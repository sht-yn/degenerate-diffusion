\documentclass[a4paper,11pt]{jsarticle}
\usepackage{yano}
\begin{document}
\section{Introduction}
Let $(\Omega,\mathcal{F},(\mathcal{F}_t)_{t\in \mathbb{R}_+ } ,P)$ be a standard stochastic basis. We will consider an $(\mathcal{F}_t)_{t\in \mathbb{R}_+ }$-adapted process $Z_t = (X_t,Y_t)$ satisfying the following stochastic differential equation (SDE):
\begin{align} \label{eq}
    dZ_t = \begin{pmatrix}
        dX_t \\ d Y_t
    \end{pmatrix} = \begin{pmatrix}
        A(Z_t,\theta_2) \\ H(Z_t,\theta_3)
    \end{pmatrix}dt + \begin{pmatrix}
        B(Z_t, \theta_1)\\O
    \end{pmatrix}dw_t,
\end{align}
where $A \in \Map(\mathbb{R}^{d_z} \times \overline{\Theta}_2; \mathbb{R}^{d_x})$, $B \in \Map(\mathbb{R}^{d_z}\times \overline{\Theta}_1 ; \mathbb{R}^{d_x} \otimes \mathbb{R}^r )$, $H\in \Map(\mathbb{R}^{d_z} \times \overline{\Theta}_3; \mathbb{R}^{d_y})$ and $w = (w_t)_{t\in \mathbb{R}_+ }$ is an $r$-dimensional Brownian motion. The sets of unknown parameters ${\Theta}_l \,(l =1,2,3)$ are open bounded subsets of $\mathbb{R}^{p_l}$, respectively. We denote the product set $\prod_{l=1}^3 \Theta_l $ by $\Theta$. We denote by $\theta$ or $\bar{\theta}$ a running index in $\Theta$ and by $\boldsymbol{\theta}=(\theta,\bar{\theta})$ a running index in $\Theta^2$. The set $\Theta^2$ is assumed to admit the Sobolev inequality holds: for all $M > 2(\sum_{l=1}^3 p_l) $, there exists $C>0$ such that
\begin{align}
 \sup_{ \boldsymbol{\theta} \in \Theta^2}|f (\boldsymbol{\theta}) |^{M} \leq C\sum_{l =0,1} \int_{\Theta^2}|\partial_ {\boldsymbol{\theta}}^l
 f(\boldsymbol{\theta})|^{M} d\boldsymbol{\theta}  \text{  for all $f(\boldsymbol{\theta}) \in C^1(\Theta^2)$.}
\end{align}Let $\theta^* = (\theta_1^*,\theta_2^*,\theta_3^*) \in \Theta$ denote the true value of $\theta = (\theta_1,\theta_2,\theta_3) \in \Theta$ to be estimated. We aim to estimate $\theta = (\theta_1,\theta_2,\theta_3)$ from the data $(Z_{t_{n,j} })_{j =0,\cdots,n}$ with $t_{n,j} = jh_n$, where $h_n$ satisfies $h_n  \to 0$ and $nh_n \to \infty$ as $n \to \infty$.  

The statistical inference of stochastic differential equation models under discrete observation schemes has been actively studied since the 1980s. \cite{PrakasaRao1983,PrakasaRao1988} demonstrated the least square estimators are asymptotically normal under the condition $nh_n^2 \to 0$ as $n \to \infty$, known as the "rapidly increasing experimental design" condition.

Notably, \cite{Florens-Zmirou1989} proposed an estimation method using the likelihood of a discretely approximated model created by the Euler-Maruyama method as a quasi-likelihood for stochastic differential equation models, which has become a standard approach for constructing quasi-likelihoods in subsequent research.

Research in this field has developed in three major directions. \cite{Florens-Zmirou1989} and \cite{Yoshida1992} showed that asymptotic normality of their estimators holds under the relaxed balance condition $nh_n^3 \to 0$, and \cite{Kessler1997} further relaxed this to $nh_n^p \to 0$ $(p\geq 3)$.

Furthermore, the theory of quasi-likelihood analysis established by \cite{yoshida2011polynomial} made it possible to construct a wide range of estimators beyond the conventional quasi-maximum likelihood estimator (QMLE), including quasi-Bayesian estimator (QBE) and to prove their asymptotic normality with moment convergence in a unified framework. Under this theoretical framework, various estimation methods have been developed under the relaxed observation scheme $nh_n^p \to 0$, including the Bayesian-type estimator by \cite{UchidaYoshida2012Adaptive} and the multi-step estimator (MSE) by \cite{KamataniUchida2014}. Their method involves creating multiple types of quasi-likelihood functions and improving estimator accuracy by using them adaptively - a technique known as adaptive inference. According to numerical experiments in \cite{KamataniUchida2014}, superior performance can be achieved in adaptive inference by constructing initial estimators through Bayesian estimation and then iteratively applying multi-step estimation methods.

However, all the literature mentioned above dealt with non-degenerate diffusion processes. Although there are many important degenerate equations, such as the harmonic oscillator and the FitzHugh-Nagumo model, their statistical inference involves fundamental difficulties compared to the non-degenerate case. For the degenerate diffusion process \eqref{eq}, the Euler-Maruyama approximation of the increment $Z_{t_{n,j}} - Z_{t_{n,j-1}}$ follows
\begin{align}
Z_{t_{n,j}} - Z_{t_{n,j-1}} \approx \begin{pmatrix}
  A(Z_{t_{n,j-1}},\theta_2) \\ H(Z_{t_{n,j-1}},\theta_3)
\end{pmatrix}h_n + \begin{pmatrix}
  B(Z_{t_{n,j-1}}, \theta_1)\\O
\end{pmatrix}(w_{t_{n,j}} - w_{t_{n,j-1}}).
\end{align}
Due to the degeneracy in the diffusion coefficient matrix, this standard Euler-Maruyama discretization fails to capture the underlying structure of the degenerate system: the distribution of this approximation is not absolutely continuous with respect to the Lebesgue measure, resulting in a singular measure. This makes it impossible to construct a quasi-likelihood function using the conventional local Gaussian approximation approach.

Recent studies \cite{ditlevsen2019hypoelliptic}, \cite{gloter2020adaptive,gloter2021adaptive,gloter2024non,gloter2024quasi}, \cite{iguchi2023parameter,iguchi2024parameter,iguchi2025parameter}, \cite{samson2025inference} have shown that this difficulty can be addressed through higher-order approximations based on the Itô-Taylor expansion, which introduces additional Gaussian terms with order $O(h_n^{3/2})$. \cite{gloter2020adaptive,gloter2021adaptive,gloter2024non} constructed adaptive and non-adaptive QMLE under the observation scheme $nh_n^2 \to 0$ by incorporating these higher-order Itô-Taylor expansion terms. Recently, \cite{iguchi2023parameter} extended the approach to construct adaptive QMLE for degenerate diffusion processes under the more general condition $nh_n^p \to 0$.

Furthermore, \cite{iguchi2025parameter} conducted Bayesian inference in the low-frequency context and \cite{samson2025inference} discusses Bayesian inference for the FitzHugh-Nagumo model. However, there has not yet been any research demonstrating asymptotic normality for methods other than QMLE for degenerate diffusion processes in the high-frequency observation context.

In this paper, we further develop the adaptive inference for degenerate SDEs under the framework of the relaxed balance condition $nh_n^p \to 0$. We provide a unified framework that allows flexible combinations of estimation methods at each step: QMLE, QBE, and MSE. We establish asymptotic normality with moment convergence of all possible hybrid estimators under this general setting through quasi-likelihood analysis.

This paper is organized as follows. First, in Section 2, we explain all necessary notations and assumptions. In Section 3,  the methods to construct our estimators and our main theorem (Theorem \ref{main}) are introduced. In Section 4, we present simulations for a linear model and the FitzHugh-Nagumo model. Section 5 and beyond are devoted to proving the main theorem. Section 5 introduce an appropriately modified version of quasi-likelihood analysis for this paper. In Section 6, we prove the main theorem using quasi-likelihood analysis assuming a central limit theorem (Proposition \ref{taihen2}) and asymptotic analyses of our quasi-likelihoods (Proposition \ref{taihen}). In section 7, we prepare basic estimations utilized in the following sections. In Section 8, we conduct the Itô-Taylor expansion for the degenerate diffusion process, and in this Section, Proposition \ref{taihen2} is proved. In Section 9, under the preparation of Section 7,8, Proposition \ref{taihen} is proved.



\end{document}